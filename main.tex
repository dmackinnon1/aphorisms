\documentclass[14pt]{extarticle}
\usepackage[utf8]{inputenc}

%use lettrine, increase the font after the initial cap
\usepackage{lettrine}
\renewcommand{\LettrineTextFont}{\large\scshape}

%set up the page numbering
\usepackage{fancyhdr}
\pagestyle{fancy}
\renewcommand{\headrulewidth}{0pt}
\fancyfoot[C]{\large{\textit{\thepage}}}
\setlength{\headheight}{17pt}

\usepackage{csquotes}
\usepackage[left=3.5cm,top=2cm, bottom=2cm, right=3.5cm]{geometry}

%specialized section title
\usepackage{titlesec}
\titleformat{\section}
{\normalfont}
{\thesection}{1em}{}
\titlespacing{\section}
{-4pc}{3.5ex plus .1ex minus .2ex}{1.5ex minus .1ex}

%put extra space between each paragraph
\setlength{\parskip}{0.75cm}

%create a command so that the lettrine parameters are set once
\newcommand{\aphor}[2]{
    \lettrine[lines=2, lraise=0.15]{#1}{#2}
}

\begin{document}

\section*{\LARGE\sffamily\slshape Aphorisms from Zen Mind, Beginner's Mind \\ {\large by SHUNRYU SUZUKI }}

\aphor{B}{eginners mind} \enquote{\textit{In the beginner's mind there are many possibilities, but in the expert's there are few.}}

\aphor{P}{osture} \enquote{\textit{These forms are not the means of obtaining the right state of mind. To take this posture is itself to have the right state of mind. There is no need to obtain some special state of mind.}}

\aphor{B}{reathing} \enquote{\textit{What we call \enquote{I} is just a swinging door which moves when we inhale and when we exhale.}}

\aphor{C}{ontrol} \enquote{\textit{To give your sheep or cow a large, spacious meadow is the way to control him.}}

\aphor{M}{ind waves} \enquote{\textit{Because we enjoy all aspects of life as an unfolding of big mind, we do not care for any excessive joy. So we have imperturbable composure.}}

\aphor{M}{ind weeds} \enquote{\textit{You should rather be grateful for the weeds you have in your mind, because eventually they will enrich your practice.}}


\aphor{T}{he marrow of Zen} \enquote{\textit{In the zazen posture, our mind and body have great power to accept things as they are, whether agreeable or disagreeable.}}


\aphor{N}{o dualism} \enquote{\textit{To stop your mind does not mean to stop the activities of mind. It means your mind pervades your whole body. With your full mnd you form the mudra in your hands.}}

\pagebreak

\aphor{N}{othing special} \enquote{\textit{If you continue this simple practice every day, you will obtain some wonderful power. Before you attain it, it is something wonderful, but after you attain it, it is nothing special.}}

\aphor{S}{ingle-minded way} \enquote{\textit{Even if the sun were to rise from the west, the Bodhisattva has only one way.}}

\aphor{R}{epetition} \enquote{\textit{If you lose the spirit of repetition, your practice will become quite difficult.}}

\aphor{Z}{en and excitement} \enquote{\textit{Zen is not some kind of excitement, but concentration on our usual everyday routine.}}

\aphor{R}{ight effort} \enquote{\textit{If your practice is good, you may become proud of it. What you do is good, but something more is added to it. Pride is extra. Right effort is to get rid of something extra.}}

\aphor{N}{o trace} \enquote{\textit{When you do something, you should burn yourself completely, like a good bonfire, leaving no trace of yourself.}}

\aphor{G}{od giving} \enquote{\textit{\enquote{To give is non-attachment,} that is, just not to attach to anything is to give.}}

\aphor{M}{istakes in practice} \enquote{\textit{It is when your practice is rather greedy that you become discouraged with it. So you should be grateful that you have a sign or warning signal to show you the weak points in your practice.}}

\aphor{L}{imiting your activity} \enquote{\textit{Usually when someone believes in a particular religion, his attitude becomes more and more a sharp angle pointing away from himself. In our way the point of the angle is always towards ourselves.}}

\pagebreak

\aphor{S}{tudy yourself} \enquote{\textit{To have some deep feeling about Buddhism is not the point; we just do what we should do, like eating supper and going to bed. This is Buddhism.}}


\aphor{T}{o polish a tile} \enquote{\textit{When you become you, Zen becomes Zen. When you are you, you see things as they are, and you become one with your surroundings.}}

\aphor{C}{onstancy} \enquote{\textit{People who know the sate of emptiness will always be able to dissolve their problems by constancy.}}

\aphor{C}{ommunication} \enquote{\textit{Without any intentional, fancy way of adjusting yourself, to express yourself as you are is the most important thing.}}

\aphor{N}{egative and positive} \enquote{\textit{Big mind is something to express, not something to figure out. Big mind is something you have, not something to seek for.}}

\aphor{N}{irvana, the waterfall} \enquote{\textit{Our life and death are the same thing. When we realize this fact, we have no fear of death anymore, nor actual difficulty in our life.}}

\aphor{T}{raditional Zen spirit} \enquote{\textit{If you are trying to attain enlightenment, you are creating and being driven by karma, and you are wasting your time on your black cushion.}}

\aphor{T}{ransiency} \enquote{\textit{We should find perfect existenxe through imperfect existence.}}

\aphor{T}{he quality of being} \enquote{\textit{When you do someting, if you fix your mind on the activity with some confidence, the quality of your state of mind is the activity itself. When you are concentrated on the quality of your being, you are prepared for the activity.}}

\pagebreak

\aphor{N}{aturalness} \enquote{\textit{Moment after moment, everyone comes out from nothingness. This is the true joy of life.}}

\aphor{E}{mptiness} \enquote{\textit{When you study Buddhism you should have a general house cleaning of your mind.}}

\aphor{R}{eadiness, mindfulness} \enquote{\textit{It is the readiness of the mind that is wisdom.}}

\aphor{B}{elieving in nothing} \enquote{\textit{In our everyday life our thinking is ninety-nine percent self-centered. \enquote{Why do I have suffering? Why do I have trouble?}}}

\aphor{A}{ttachment, non-attachment} \enquote{\textit{That we are attached to some beauty is also Buddha's activity.}}

\aphor{C}{almness} \enquote{\textit{For Zen students, a weed is a treasure.}}

\aphor{E}{xperience, not philosophy} \enquote{\textit{There is something blasphemous in talking about how Buddhism is perfect as a philosophy or teaching without knowing what it actually is.}}

\aphor{B}{eyond consciousness} \enquote{\textit{To realize pure mind in your delusion is practice. If you try to expel the delusion it will oly persist the more. Just say, \enquote{Oh, this is just delusion.} and do not be bothered by it.}}

\aphor{B}{uddha's enlightenment} \enquote{\textit{If you take pride in your attainment or become discouraged because of your idealistic effort, your practice will confine you by a thick wall.}}

\aphor{Z}{en mind} \enquote{\textit{Before the rain stops we can hear a bird. Even under the heavy snow we see snowdrops and some new growth.}}



\end{document}
